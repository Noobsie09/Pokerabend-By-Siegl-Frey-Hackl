\documentclass[a4paper,12pt]{article}

\usepackage[utf8]{inputenc}
\usepackage[T1]{fontenc}
\usepackage[german]{babel}
\usepackage{graphicx}
\usepackage{geometry}
\usepackage[hidelinks]{hyperref}
\usepackage{enumitem}
\usepackage{titlesec}
\usepackage{float}

\geometry{margin=2.5cm}

\begin{document}

\begin{titlepage}
    \centering
    \vspace*{2cm}
    
    {\Huge \textbf{Projektarbeit: Planung und Durchführung eines Pokerabends}}\\[1.5cm]

    \begin{figure}[H]
        \centering
        \includegraphics[width=0.55\textwidth]{poker.jpeg}\\[1.5cm]
        \caption{Poker}
        \label{fig:Poker}
    \end{figure}
    
    
    
    {\Large \textbf{„Poker Night – Planung \& Organisation eines Spieleabends“}}\\[1.5cm]
    
    \textbf{Projektteam:}\\
    Projektleiter: Rafael Crispin Siegl\\
    Projektmitarbeiter: Noel Frey\\
    Projektmitarbeiter: Simon Hackl\\[1cm]
    
    \textbf{Projektauftraggeber:}\\
    Herr Prof. Neuner\\[0.5cm]
    
    \textbf{Kunde:}\\
    Teilnehmerinnen und Teilnehmer des Pokerabends\\[0.5cm]
    
    \textbf{Abgabe:}\\
    15. Dezember
    
    \vfill
\end{titlepage}

\tableofcontents
\listoffigures

\newpage

\section{Projektbeschreibung}

Unser Projekt beschäftigt sich mit der vollständigen Planung und Umsetzung eines Pokerabends. Dabei sollen alle notwendigen organisatorischen Schritte – von der Einladung über die Vorbereitung der Spielmaterialien bis hin zur Verpflegung – sorgfältig durchdacht und umgesetzt werden.

Ziel ist es, einen \textbf{gemütlichen, unterhaltsamen und strukturierten Abend} zu gestalten, an dem alle Teilnehmenden Spaß haben und ein reibungsloser Ablauf gewährleistet ist.

\newpage

\section{Zielsetzung}

Wir wollen bis spätestens \textbf{15. Dezember} einen \textbf{Pokerabend für mindestens zehn Personen} veranstalten.  
Das Gesamtbudget beträgt \textbf{maximal 100 €}.

Es soll sichergestellt werden, dass:

\begin{itemize}
    \item ausreichend \textbf{Spielkarten}, \textbf{Pokerchips} und \textbf{Tische} vorhanden sind
    \item genügend \textbf{Snacks und Getränke} zur Verfügung stehen
    \item alle Teilnehmer \textbf{rechtzeitig informiert} werden
    \item der Ablauf klar geplant ist
    \item Aufgaben sinnvoll verteilt werden, um einen \textbf{entspannten und erfolgreichen Abend} zu ermöglichen
\end{itemize}

\section{Projektorganisation}

\subsection*{Projektauftraggeber}
Herr Professor Neuner

\subsection*{Projektleiter}
\textbf{Rafael Crispin Siegl} \\
Verantwortlich für Planung, Terminkoordination und den Überblick über alle Arbeitsbereiche.

\subsection*{Projektmitarbeiter}
\begin{itemize}
    \item Noel Frey → Organisation \& Kommunikation
    \item Simon Hackl → Snacks \& Dekoration
\end{itemize}

\subsection*{Kunde}
Teilnehmerinnen und Teilnehmer des Pokerabends

\newpage

\section{Organisationsform: Matrix-Organisation}

Wir haben uns für eine \textbf{Matrix-Organisation} entschieden.

Das bedeutet:

\begin{itemize}
    \item Zusammenarbeit erfolgt \textbf{eng und flexibel}
    \item Aufgaben werden nach \textbf{Zeitkapazität und Fähigkeiten} verteilt
    \item Der Projektleiter behält die Übersicht
    \item Teammitglieder unterstützen sich \textbf{gegenseitig}
    \item Die Struktur hilft uns, das Projekt effizient und zielgerichtet zum Erfolg zu führen
\end{itemize}

\newpage

\section{Projektstrukturplan (PSP)}

Hier wird der Projektstrukturplan dargestellt. Er gliedert das Projekt in übersichtliche Arbeitspakete und Teilbereiche.

\begin{figure}[H]
    \centering
    \includegraphics[width=\textwidth]{projektstrukturplan.png}
    \caption{Projekstrukturplan}
    \label{fig:Projekstrukturplan}
\end{figure}

\section{Gantt-Chart}

Das folgende Gantt-Chart zeigt die zeitliche Planung des Projekts.

\begin{figure}[H]
    \centering
    \includegraphics[width=\textwidth]{GANTT-Diagramm.png}
    \caption{GANTT-Diagramm}
    \label{fig:GANTT-Diagramm}
\end{figure}

\newpage

\section{Projektumfeldanalyse}

\begin{figure}[H]
    \centering
    \includegraphics[width=\textwidth]{Projektumfeldanalyse.png}
    \caption{Projektumfeldanalyse}
    \label{fig:projektumfeldanalyse}
\end{figure}
\begin{figure}[H]
    \centering
    \includegraphics[width=\textwidth]{Diagramm.png}
    \caption{Projektumfeldanalyse grafisch}
    \label{fig:projektumfeldanalyse grafisch}
\end{figure}

\newpage

\section{Risikoanalyse}

\begin{figure}[H]
    \centering
    \includegraphics[width=\textwidth]{Risikoportfolio.png}
    \caption{Risikoportfolio}
    \label{fig:risikoportfolio}
\end{figure}

\begin{figure}[H]
    \centering
    \includegraphics[width=\textwidth]{Risikomatrix.png}
    \caption{Risikomatrix}
    \label{fig:risikomatrix}
\end{figure}

\begin{figure}[H]
    \centering
    \includegraphics[width=\textwidth]{Heatmap.png}
    \caption{Heatmap}
    \label{fig:heatmap}
\end{figure}

\newpage

\section{User Stories}

\subsection{Pokerabend organisieren (Gastgeber)}

\textbf{Als Gastgeber} möchte ich einen festen Termin und Ort für den Pokerabend festlegen,  
\textbf{damit} alle Freunde rechtzeitig planen können.

\subsubsection*{Akzeptanzkriterien}
\begin{itemize}
    \item Termin und Ort sind mindestens \textbf{eine Woche im Voraus} festgelegt.
    \item Alle eingeladenen Freunde wurden über \textbf{Datum, Uhrzeit und Adresse} informiert.
    \item Mindestens \textbf{75 \% der Eingeladenen} haben ihre Teilnahme bestätigt.
\end{itemize}

\subsection{Spielregeln festlegen (Mitspieler)}

\textbf{Als Mitspieler} möchte ich die Spielregeln vorab kennen,  
\textbf{damit} es während des Pokerabends keine Missverständnisse gibt.

\subsubsection*{Akzeptanzkriterien}
\begin{itemize}
    \item Die Spielvariante (z.\,B. Texas Hold’em) ist klar definiert.
    \item Einsatzhöhe und Buy-in sind vorab kommuniziert.
    \item Die Blind-Struktur ist festgelegt und verständlich dokumentiert.
    \item Alle Teilnehmer bestätigen, dass sie die Regeln verstanden haben.
\end{itemize}

\subsection{Snacks \& Getränke organisieren (Teilnehmer)}

\textbf{Als Teilnehmer} möchte ich Snacks und Getränke gemeinsam abstimmen oder aufteilen,  
\textbf{damit} sich alle wohlfühlen und niemand alles allein organisieren muss.

\subsubsection*{Akzeptanzkriterien}
\begin{itemize}
    \item Es existiert eine Liste mit Snacks und Getränken.
    \item Jeder Teilnehmer übernimmt mindestens \textbf{einen Beitrag}.
    \item Besondere Vorlieben oder Unverträglichkeiten sind berücksichtigt.
    \item Zum Start des Pokerabends sind alle zugesagten Snacks und Getränke vorhanden.
\end{itemize}

\end{document}
